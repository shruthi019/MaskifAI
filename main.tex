\documentclass[14pt]{beamer}


\title{MASKIFY}
\subtitle{Jun-July 2020 Project}
\author[TEAM 6]{Sachita Malhotra, Shruthi Rao, Srishti Negi}
\date{June 2020}


\usetheme{AnnArbor}

\newcounter{saveenumerate}
\newcommand{\saveenumerate}{\setcounter{saveenumerate}{\value{enumi}}}
\newcommand{\restartenumerate}{\setcounter{enumi}{\value{saveenumerate}}}

\begin{document}

\begin{frame}
    \titlepage
\end{frame}

\begin{frame}
    \frametitle{Overview}
    Our project aims at creating a mask detection web app.
\end{frame}

\begin{frame}
    \frametitle{Tech stack}
    \begin{enumerate}
        \item OpenCV - Will be used for mask detection.
        \item Flask/Django - Will be used for creation of web app.
    \end{enumerate}
\end{frame} 

\begin{frame}
    \frametitle{Description}
    \begin{enumerate}
        \item The web app will be installed in a system at the office entrance.

        \item The employees will log into the system and use the application to confirm that they are wearing a mask properly.

        \item In case the above condition is not satisfied it will notify their manager about those employees.
        \saveenumerate
    \end{enumerate}
\end{frame}

\begin{frame}
    \frametitle{Description}
    \begin{enumerate}
        \restartenumerate
        \item The application will also send periodical sanitization notification to the employees.
            
        \item It will provide general information about maintaining proper hygiene and precautions for Covid-19 to the user if ze logs into the application in their own systems.
        
        \item The application will contain a user database of the employees wherein it will record this data.
        
        \item The manager will be able to access this data.
    \end{enumerate}
\end{frame}

\begin{frame}
    \frametitle{Target Audience}
    The web application is  aimed at helping organizations take precautions against Covid-19 by ensuring that all the employees who enter the workspace wear masks.
\end{frame}

\end{document}
